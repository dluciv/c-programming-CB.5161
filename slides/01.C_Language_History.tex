\documentclass[xetex,aspectratio=149]{beamer}

\usepackage{res/lections}

\preamble

\title[C History]{Brief Reasoning and History of~the~С~Programming~Language}

\begin{document}

\titleslide

\tocslide

\section{From the Author}

\begin{frame}{Dennis Ritchie: The Development of the C Language}
	\url{https://www.bell-labs.com/usr/dmr/www/chist.html}
	\begin{itemize}
		\item Setting
		\item Origins in Other Languages
		\item Later Usage
	\end{itemize}
	We will address to it
\end{frame}

\section{Reasoning}

\begin{frame}{Computer Generations}
	\begin{enumerate}
		\item 1940s--1950s. Relays and vacuum tubes: $10^5$ watts, many rooms; available for military purposes (and then for other physical computations)
		\item 1950s--1960s. Semiconductors (transistors, diodes): $10^4$ watts, several racks; available for large institutions, banks
		\item 1960s--1970s. Integrated circuits: $10^2-10^3$ watts, one or several racks; available for smaller institutions and laboratories
		\item 1970s--1980s--now. Microporcessors in single integrated circuit: $10-10^2$ watts,
		one box, available for any organization and later for personal use
	\end{enumerate}
\end{frame}

\begin{frame}{Setting}
	\begin{block}{Common approach of 1960s}
		\begin{itemize}
			\item Mainframes like IBM/360 or GE-645
			\item Programming languages like PL/I
			\item Operating systemc like OS/360 or Multics
			\item Batch control approach like JCL
			\item No powerful interactive shell
		\end{itemize}
		Everything is complicated and heavy-weight
	\end{block}
\pause
	\begin{block}{New approach of 1970s}
	\begin{itemize}
		\item Simpler \emph{and cheaper} mini-computers like DEC PDP-7
		\item More universal use of them
		\item Many computer families
	\end{itemize}
\end{block}
\end{frame}

\begin{frame}{Birth of Unix}
Killer-features of Unix:

\begin{itemize}
	\item Hierarchical file system with single tree of file names
	\item Agnostic approach to file data: before, data was usually stored in formatted files, which offered good throughput but were complicated for software developers
	\item Interactive powerful shell running in user space
	\pause
	\item And one more feature that we will describe later\ldots
\end{itemize}

\end{frame}

\begin{frame}{Before and After Unix}

Note:

\begin{itemize}
	\item Multics already offered many of above features, but still was too complicated; minimalist design was desired
	\item Above approaches were very good finding and they are still actual after 50 years: we see elements of such a design in such OSs as DOS and then Windows
\end{itemize}

See a pretty nice AT\&T documentary on this:
\begin{itemize}
	\item \url{https://youtu.be/tc4ROCJYbm0}
\end{itemize}

\end{frame}

\section{Early History}

\begin{frame}{Popular languages of 1960s and before}
	\begin{itemize}
		\item Fortran: one of the first, high-level, computational
		\item COBOL: business-oriented language
		\item PL/I: general purpose complicated language, suited better for systems programming than above two
		\item Assembly languages for many computer architectures, not portable
	\end{itemize}

\pause

	\begin{itemize}
		\item \emph{All above --- not only assembly ---} were not very portable
		\item They were not structural languages, which lead to poor quality code which was difficult to maintain due to spaghetti-code
	\end{itemize}
\end{frame}

\begin{frame}{Birth of C}
	\begin{itemize}
		\item Compiled language
		\item Structural language
		\item Good for systems programming
		\item Simple enough and portable
	\end{itemize}
\end{frame}

\begin{frame}{What is structural programming?}
	Program consists of:
	\begin{itemize}
		\item Sequential blocks of operators
		\item Loops
		\item Branching (\mintinline{C}|if| --- \mintinline{C}|else| --- \ldots)
		\item All above can be used withing each other
	\end{itemize}
	\pause

	\emph{B\"{o}hm-Jecopini theorem}: above are enough to express any algorithm in sense of Turing-completeness
	\pause
	
	Additionally:
	\begin{itemize}
		\item \mintinline{C}|goto| is available but not welcome
		\item Procedures!
		\item Clean variable scopes (not as in Basic or Python!)
	\end{itemize}
\end{frame}

\begin{frame}{More links to look at}
	\begin{itemize}
		\item Notes on Structured Programming. By Prof. Dr. Edsger W. Dijkstra — T. H. Report 70-WSK-03 Second Edition April 1970
		\item Dijkstra: EWD 215: A Case against the GO TO Statement (PDF).
	\end{itemize}
\end{frame}

\begin{frame}{C Predecessors}
	\begin{itemize}
		\item 1960: \href{https://en.wikipedia.org/wiki/ALGOL_60\#Code_sample_comparisons}{Algol-60}
		\item 1963: \href{https://en.wikipedia.org/wiki/CPL_(programming_language)\#Example}{CPL}
		\item 1967: \href{https://en.wikipedia.org/wiki/BCPL\#Examples}{BCPL}
		\item 1969: \href{https://en.wikipedia.org/wiki/B_(programming_language)\#Examples}{B}
		\item 1972: \href{https://en.wikipedia.org/wiki/C_(programming_language)\#\%22Hello,_world\%22_example}{C}
	\end{itemize}
\end{frame}

\section{With Unix}

\begin{frame}{C and Unix evolved together}
	\begin{itemize}
		\item C was portable (not referring any particular architecture properties)
		\item C was simple enough to create new compiler targets quickly
		\item C suited well for systems programming
	\end{itemize}

	\pause
	In beginning of 1970s the majority of Unix code was re-implemented in C, which was one of the reasons of its popularity till now. Now we have it in servers, networking hardware, PCs, mobiles etc.
\end{frame}

\section{Recent History}

\begin{frame}{1980s--1990s}
	\begin{itemize}
	\item Cheap PCs
	\item Internet
	\end{itemize}
\end{frame}

\begin{frame}{1990s--2000s--now}
	\begin{itemize}
		\item Many mobile and embedded architectures
		\item Parallel architectures
	\end{itemize}
\end{frame}

\postamble

\end{document}
